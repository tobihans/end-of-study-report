\conclusion
L'évolution des méthodes d'enseignement implique l'utilisation croissante des technologies de l'information et de la communication. 
Cette tendance est motivée par la nécessité de s'adapter aux nouveaux besoins d'apprentissage et de compenser les insuffisances logistiques et 
financières auxquelles font face les établissements d'enseignement supérieur. Dans ce contexte, 
le projet auquel nous avons contribué visait à fournir un cadre moderne de communication en temps réel, 
permettant d'explorer de nouvelles possibilités en matière de formation virtuelle. 
Nous sommes convaincus que l'application que nous avons développée ouvre la voie à une large gamme d'opportunités en matière d'exploitation des classes virtuelles, 
et nous espérons que notre travail sera une contribution utile à l'ensemble de la communauté éducative.  
Bien qu'ayant atteint ses objectifs initiaux et répondu aux besoins identifiés, il reste encore de nombreuses possibilités d'extension pour en faire un outil encore plus puissant et adapté aux besoins évolutifs de l'enseignement en ligne. 
Il est donc envisageable d'explorer des pistes de développement pour étendre l'application au-delà de ses fonctions actuelles.