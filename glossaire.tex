% WebRTC related

\newacronym{webrtc}{WebRTC}{Web Real-Time Communication}

\newglossaryentry{udp}{type=\acronymtype, name={UDP}, description={User Datagram Protocol\glsadd{udp_def}},see=[:]{udp_def}}
\newglossaryentry{udp_def}
{
  name=UDP,
  description={Protocole de transmission de données sans connexion (sous forme de datagrammes) entre deux entités.}
}

\newglossaryentry{ip}{type=\acronymtype, name={IP}, description={Internet Protocol\glsadd{ip_def}},see=[:]{ip_def}}
\newglossaryentry{ip_def}
{
  name=IP,
  description={Famille de protocoles de communication de réseaux informatiques conçus pour être utilisés sur Internet.}
}

\newglossaryentry{p2p}{type=\acronymtype, name=P2P, description={Peer To Peer\glsadd{p2p_def}},see=[:]{p2p_def}}
\newglossaryentry{p2p_def}
{
  name=P2P,
  description={Modèle d'échange en réseau où chaque entité est à la fois client et serveur.}
}

\newglossaryentry{ice}{type=\acronymtype, name=ICE, description={Interactive Connectivity Establishment\glsadd{ice_def}}, see=[:]{ice_def}}
\newglossaryentry{ice_def}
{
  name=ICE,
  description={Protocole utilisé dans les réseaux informatiques pour trouver des moyens permettant à deux ordinateurs de se parler aussi directement que possible dans le cadre d'un réseau \gls{p2p}.}
}

\newglossaryentry{sdp}{type=\acronymtype, name=SDP, description={Session Description Protocol\glsadd{sdp_def}}, see=[:]{sdp_def}}
\newglossaryentry{sdp_def}
{
  name=SDP,
  description={Protocole de communication de description de paramètres d'initialisation d'une session de diffusion en flux.}
}

\newglossaryentry{nat}{type=\acronymtype, name=NAT, description={Network Address Translation\glsadd{nat_def}}, see=[:]{nat_def}}
\newglossaryentry{nat_def}
{
  name=NAT,
  description={Méthode de conversion d'un espace d'adresses \gls{ip} en un autre en modifiant les informations relatives aux adresses de réseau dans l'en-tête \gls{ip} des paquets pendant qu'ils transitent par un dispositif d'acheminement du trafic.}
}

\newglossaryentry{dtls}{type=\acronymtype, name=DTLS, description={Datagram Transport Layer Security\glsadd{dtls_def}}, see=[:]{dtls_def}}
\newglossaryentry{dtls_def}
{
  name=DTLS,
  description={Protocole qui fournit une sécurisation des échanges basés sur des protocoles en mode datagramme.}
}

\newglossaryentry{srtp}{type=\acronymtype, name=SRTP, description={Secure Real-Time Transport Protocol\glsadd{srtp_def}}, see=[:]{srtp_def}}
\newglossaryentry{srtp_def}
{
  name=SRTP,
  description={Protocole employant le cryptage et l'authentification pour minimiser le risque d'attaques par déni de service et les failles de sécurité du protocole \gls{rtp}.}
}

\newglossaryentry{stun}{type=\acronymtype, name=STUN, description={Session Traversal Utilities for NAT\glsadd{stun_def}}, see=[:]{stun_def}}
\newglossaryentry{stun_def}
{
  name=STUN,
  description={Protocole client-serveur permettant à un client \gls{udp} situé derrière un routeur \gls{nat} de découvrir son adresse \gls{ip} publique ainsi que le type de routeur \gls{nat} derrière lequel il est.}
}

\newglossaryentry{turn}{type=\acronymtype, name=TURN, description={Traversal Using Relays around NAT\glsadd{turn_def}}, see=[:]{turn_def}}
\newglossaryentry{turn_def}
{
  name=TURN,
  description={Protocole qui aide à traverser les traducteurs d'adresses réseau ou les pare-feu pour les applications multimédias.}
}

\newglossaryentry{sctp}{type=\acronymtype, name=SCTP, description={Stream Control Transmission Protocol\glsadd{sctp_def}}, see=[:]{sctp_def}}
\newglossaryentry{sctp_def}
{
  name=SCTP,
  description={Protocole de transmission fiable des messages de télécommunication à travers des réseaux \gls{ip}.}
}

\newglossaryentry{rtp}{type=\acronymtype, name=RTP, description={Real-time Transport Protocol\glsadd{rtp_def}}, see=[:]{rtp_def}}
\newglossaryentry{rtp_def}
{
  name=RTP,
  description={Protocole réseau qui décrit comment transmettre divers médias (audio, vidéo) d'un point de terminaison à un autre en temps réel. }
}

\newglossaryentry{sfu}{type=\acronymtype, name=SFU, description={Selective Forward Unit\glsadd{sfu_def}}, see=[:]{sfu_def}}
\newglossaryentry{sfu_def}
{
  name=SFU,
  description={Serveur qui relaie le trafic multimédia, où chaque homologue s'y connecte pour récupérer les flux de média.}
}


% Courses related
\newacronym{mooc}{MOOC}{Massive Open Online Courses}

\newacronym{smoc}{SMOC}{Synchronous Massive Online Courses}

\newacronym{spoc}{SPOC}{Small Private Online Courses}

\newacronym{ssoc}{SSOC}{Synchronous Small Online Courses}


% Software Modeling
\newacronym{omg}{OMG}{Object Management Group}
\newacronym{uml}{UML}{Unified Modeling Language}

% Programming
\newacronym{api}{API}{Application Programming Interface}
\newacronym{rest}{REST}{Representational State Transfer}
\newacronym{http}{HTTP}{HyperText Transfer Protocol}
\newacronym{html}{HTML}{HyperText Markup Language}
\newacronym{css}{CSS}{Cascading StyleSheet}
\newacronym{pwa}{PWA}{Progressive Web Appps}
\newacronym{tls}{TLS}{Transport Layer Security}
