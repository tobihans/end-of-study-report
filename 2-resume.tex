\resume
\selectlanguage{french}
\vspace*{-6cm}
\begin{abstract}
L'enseignement supérieur est confronté à de nouvelles contraintes, telles que l'impossibilité d'assister aux cours en présentiel en raison de divers facteurs comme les pandémies. 
Les technologies de communication en temps réel offrent une alternative prometteuse, qu'il convient d'exploiter pour répondre à ces défis. 
Afin de favoriser l'utilisation de ces outils de communication, notre projet propose StudX, un prototype d'application permettant la tenue de classes virtuelles. 
Le développement de cette plateforme a nécessité l'utilisation de techniques modernes de modélisation logicielle, de technologies de pointe et de ressources variées. 
StudX offre une solution fiable pour répondre aux besoins des universités en matière d'enseignement à distance, grâce à des fonctionnalités de communication en temps réel basées sur WebRTC. 
Ce projet est une contribution importante à l'exploitation des technologies de communication en temps réel dans le domaine de l'éducation et offre des perspectives d'extension pour répondre à de nouveaux besoins.\newline\newline
\textbf{Mots clés}: StudX, WebRTC, cours en ligne, universités, communication en temps réel
\end{abstract}

\newpage
\thispagestyle{empty}
\selectlanguage{english}
\addcontentsline{toc}{chapter}{Abstract}
\begin{abstract}

L'enseignement supérieur est confronté à de nouvelles contraintes, telles que l'impossibilité d'assister aux cours en présentiel en raison de divers facteurs comme les pandémies. 
Les technologies de communication en temps réel offrent une alternative prometteuse, qu'il convient d'exploiter pour répondre à ces défis. 
Afin de favoriser l'utilisation de ces outils de communication, notre projet propose StudX, un prototype d'application permettant la tenue de classes virtuelles. 
Le développement de cette plateforme a nécessité l'utilisation de techniques modernes de modélisation logicielle, de technologies de pointe et de ressources variées. 
StudX offre une solution fiable pour répondre aux besoins des universités en matière d'enseignement à distance, grâce à des fonctionnalités de communication en temps réel basées sur WebRTC. 
Ce projet est une contribution importante à l'exploitation des technologies de communication en temps réel dans le domaine de l'éducation et offre des perspectives d'extension pour répondre à de nouveaux besoins.\newline\newline
\textbf{Keywords}: StudX, WebRTC, online classes, universities, real time communication
\end{abstract}