\resume
\selectlanguage{french}
\vspace*{-6cm}
\begin{abstract}
L'enseignement supérieur est confronté à de nouvelles contraintes ; en témoigne la pandémie de COVID-19 qui a pertubé le déroulement
de cours présentiel. Les technologies de communication en temps réel offrent une alternative prometteuse, qu'il convient d'exploiter pour répondre à ces défis. 
Afin de favoriser l'utilisation de ces outils de communication, notre projet propose StudX, un prototype d'application permettant la tenue de classes virtuelles. 
Le développement de cette plateforme a nécessité l'utilisation de techniques modernes de modélisation logicielle, de technologies de pointe et de ressources variées. 
StudX offre une solution fiable pour répondre aux besoins des universités en matière d'enseignement à distance, grâce à des fonctionnalités de communication en temps réel basées sur WebRTC. 
Ce projet est une contribution importante à l'exploitation des technologies de communication en temps réel dans le domaine de l'éducation et offre des perspectives d'extension pour répondre à de nouveaux besoins.\newline\newline
\textbf{Mots clés}: StudX, WebRTC, cours en ligne, universités, communication en temps réel
\end{abstract}

\newpage
\thispagestyle{empty}
\selectlanguage{english}
\addcontentsline{toc}{chapter}{Abstract}
\begin{abstract}

Higher education is facing new constraints, such as the inability to attend face-to-face classes due to various factors such as pandemics. 
Real-time communication technologies offer a promising alternative that must be exploited to meet these challenges. 
In order to promote the use of these communication tools, our project proposes StudX, a prototype application for virtual classes. 
The development of this platform required the use of modern software modeling techniques, state-of-the-art technologies and various resources. 
StudX provides a reliable solution to the distance learning needs of universities through WebRTC-based real-time communication capabilities. 
This project is an important contribution to the exploitation of real-time communication technologies in the field of education and has the potential to be extended to meet new needs.\newline\newline
\textbf{Keywords}: StudX, WebRTC, online classes, universities, real time communication
\end{abstract}