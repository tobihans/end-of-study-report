\resume
\selectlanguage{french}
\vspace*{-6cm}
\begin{abstract}
L'enseignement supérieur doit s'adapter aux nouvelles contraintes que l'on observe aujourd'hui comme l'incapacité
d'assiter aux cours presentiels en raison de facteurs comme les pandémies. Les technologies de communication en temps
réel offrent une bonne alternative qu' il convient d'exploiter. Dans le but de favoriser l'emploi de ces outils de communication,
nous proposons StudX, un prototype d'application permettant la tenue de classes virtuelles. Le processus de 
développement de cette plateforme a requis l'emploi de techniques modernes de modélisation logicielle, de technologies de pointe et de ressources diverses.\newline

\textbf{Mots clés}: StudX, WebRTC, cours en ligne, universités, communication en temps réel
\end{abstract}

\newpage
\thispagestyle{empty}
\selectlanguage{english}
\addcontentsline{toc}{chapter}{Abstract}
\begin{abstract}

Higher education must adapt to the new constraints that we observe today such as the inability to attend
to attend face-to-face classes due to factors such as pandemics. Real-time communication technologies offer a good alternative
technologies offer a good alternative that is worth exploiting. In order to promote the use of these communication tools
we propose StudX, a prototype application for virtual classes. The development process of this 
The development process of this platform required the use of modern software modeling techniques, state-of-the-art technologies and various resources.\newline

\textbf{Keywords}: StudX, WebRTC, online classes, universities, real time communication
\end{abstract}