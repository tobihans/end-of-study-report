\documentclass{beamer}

\mode<presentation> {
\usetheme{metropolis}
}

\usepackage{graphicx}
\usepackage{svg}
\usepackage{booktabs}
\usepackage{amsmath}
\usepackage[french]{babel}
\usepackage{url,color}
\usepackage{subfigure}
\usepackage{amsthm,amsfonts,amssymb,amscd,amsxtra, multicol}
\usepackage{comment}

\graphicspath{ {./images/} }

% \setbeamertemplate{caption}[numbered]
% \newcommand{\tabitem}{~~\llap{\textbullet}~~}

\title{Conception d'un SaaS pour les cours en ligne en milieu universitaire}
\author[Hans TOGNON]{Hans B. K. \textbf{TOGNON} \\ Supervisé par Ing. Miranda \textbf{GNONLONFOUN}}
\institute[IFRI]{
\textbf{I}nstitut de \textbf{F}ormation et de \textbf{R}echerche en  \textbf{I}nformatique \\
\medskip
% \textbf{\color{magenta}\href{mailto:contact@ifri.uac.bj}{contact@ifri.uac.bj}}
}

\begin{document}
% %%%%%%%%%%%%%%%%%%%%%%%%%%%%%%%%%%
% Title Page
% %%%%%%%%%%%%%%%%%%%%%%%%%%%%%%%%%%
\begin{frame}
  \thispagestyle{empty}
  \begin{multicols}{2}
    \begin{figure}
        \flushleft
        \includegraphics[width=0.11\textwidth]{logoifri}
    \end{figure}
    \begin{figure}
        \flushright
        \includegraphics[width=0.1\textwidth]{logouac}
    \end{figure}
    \end{multicols}
    \vspace{-1cm}
  \titlepage
  \end{frame}

% %%%%%%%%%%%%%%%%%%%%%%%%%%%%%%%%%%
% ToC
% %%%%%%%%%%%%%%%%%%%%%%%%%%%%%%%%%%
\begin{frame}
  \frametitle{PLAN}
  \begin{itemize}
    \item Introduction
    \item Revue de Littérature
    \item Matériels et méthodes
    \item Prototype
    \item Résultats et Perspectives
    \item Conclusion
  \end{itemize}
\end{frame}

% %%%%%%%%%%%%%%%%%%%%%%%%%%%%%%%%%%
% Introduction
% %%%%%%%%%%%%%%%%%%%%%%%%%%%%%%%%%%
\begin{frame}
  \begin{center}
    \section{\huge{Introduction}}
  \end{center}
\end{frame}

\begin{frame}{Introduction : \small{Contexte}}
  Les méthodes d'enseignement évoluent avec l'utilisation croissante 
  du numérique pour une éducation plus créative et collaborative, y 
  compris dans l'enseignement supérieur, notamment en réponse aux défis 
  tels que la pandémie de COVID-19 et l'indisponibilité de cadres de cours 
  adéquats.
\end{frame}

\begin{frame}{Introduction : \small{Problématique}}
  L'expansion des cours en ligne nécessite une organisation logistique 
  accrue et un investissement financier. Les solutions génériques s'avèrent dans bien 
  des cas inadequates ou coûteuses.
\end{frame}


\begin{frame}
  \frametitle{Introduction : \small{Objectifs}}
    Concevoir un prototype d'application permettant la tenue de cours en ligne.
\end{frame}

\begin{frame}
  \frametitle{Introduction : \small{Objectifs} - \footnotesize{Fonctionnalités}}
  \begin{itemize}
    \item Organiser les classes des entités en sections bien définies ;
    \item Définir l'organisation temporelle des classes ;
    \item Organiser des sessions d’audio-conférence ;
\end{itemize}
\end{frame}

\begin{frame}
  \frametitle{Introduction : \small{Objectifs} - \footnotesize{Fonctionnalités}}
  \begin{itemize}
    \item Emuler, un tant soit peu, un environnement de classe présentiel via les fonctionnalités intégrées ;
    \item Minimiser les coûts requis dans le cadre de la mise en oeuvre d’une solution de classe virtuelle.
\end{itemize}
\end{frame}

% %%%%%%%%%%%%%%%%%%%%%%%%%%%%%%%%%%
% Revue de Littérature
% %%%%%%%%%%%%%%%%%%%%%%%%%%%%%%%%%%
\begin{frame}
  \begin{center}
    \section{\huge{Revue de Littérature}}
  \end{center}
\end{frame}

\begin{frame}{Revue de Littérature : \small{Formation à distance}}
  \begin{description}
    \item[API] Application Programming Interface
    \item[LAN] Local Area Network
    \item[ASCII] American Standard Code for Information Interchange
    \end{description}
\end{frame}

% %%%%%%%%%%%%%%%%%%%%%%%%%%%%%%%%%%
% Matériels et méthodes
% %%%%%%%%%%%%%%%%%%%%%%%%%%%%%%%%%%
\begin{frame}
  \begin{center}
    \section{\huge{Matériels et méthodes}}
  \end{center}
\end{frame}

% %%%%%%%%%%%%%%%%%%%%%%%%%%%%%%%%%%
% Prototype
% %%%%%%%%%%%%%%%%%%%%%%%%%%%%%%%%%%
\begin{frame}
  \begin{center}
    \section{\huge{Prototype}}
  \end{center}
\end{frame}


% %%%%%%%%%%%%%%%%%%%%%%%%%%%%%%%%%%
% Résultats et Perspectives
% %%%%%%%%%%%%%%%%%%%%%%%%%%%%%%%%%%
\begin{frame}
  \begin{center}
    \section{\huge{Résultats et Perspectives}}
  \end{center}
\end{frame}



% %%%%%%%%%%%%%%%%%%%%%%%%%%%%%%%%%%
% Conclusion
% %%%%%%%%%%%%%%%%%%%%%%%%%%%%%%%%%%
\begin{frame}
  \begin{center}
    \section{\huge{Conclusion}}
  \end{center}
\end{frame}


\end{document}
