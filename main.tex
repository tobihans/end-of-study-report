\documentclass{ifri}
\usepackage{titletoc}
\setlength{\glsdescwidth}{0.65\textwidth}
% \usepackage{lscape}

\typeMemoire{Diplôme de Licence en Informatique}
\optionFormation{Génie Logiciel}
\etudiant{Hans Bignon. K. \textbf{TOGNON}}
\titreDuMemoire{--} %Implémention pour une meilleure sécurité dans les réseau LAN sous IPv6 // Proposition: Identification des vulnérabilités dans un reseau LAN IPv6 et mesures pour une meilleure sécurité.

\dateSoutenance{-}
%\promo{2\up{ème}}
\anneeScolaire{--}


%%maitre de mémoire
\encadrants{Prenom \textbf{Nom}}

%% Membres du Jury
\jurys{%
\begin{tabular}{llll}
	Nom et prénoms du président & Grade & Entité & Président \\
	Nom et prénoms de l'examinateur & Grade & Entité & Examinateur \\
	Nom et prénoms du rapporteur & Grade & Entité & Rapporteur \\
\end{tabular}	
}


\hypersetup{
 pdftitle={--},
 pdfauthor={--},
 pdfsubject={--},
 pdfkeywords={--}
 }

\color{bookColor}

%importation du glossaire
\loadglsentries{glossaire_reduit}

\begin{document}

\pageDeGarde
%\pageTitre

\pagecolor{white}

%% page vide
%\thispagestyle{empty}\ \clearpage


\selectlanguage{french}

% sommaire
\pagenumbering{roman}

\setcounter{tocdepth}{0}
\startlist{toc}
\printlist{toc}{}{\chapter*{Sommaire}}
\setcounter{tocdepth}{5}

%% rdedicaces
\dedicace
Mes dedicaces

\newpage 

%% remerciements
\remerciements

Nos remerciements
\newpage 

% Résume
\resume
\selectlanguage{french}
\vspace*{-6cm}
\begin{abstract}
L'enseignement supérieur doit s'adapter aux nouvelles contraintes que l'on observe aujourd'hui comme l'incapacité
d'assiter aux cours presentiels en raison de facteurs comme les pandémies. Les technologies de communication en temps
réel offrent une bonne alternative qui revient d'exploiter. Dans le but de favoriser l'emploi de ces outils de communication,
nous proposons StudX, un prototype d'application permettant la tenue de classes virtuelles. Le processus de 
développement de cette plateforme a requis l'emploi de techniques modernes de modélisation logicielle, de technologies de pointe et de ressources appropriées.

\paragraph{}
\textbf{Mots clés}: StudX,WebRTC,cours en ligne,universités,communication en temps réel
\end{abstract}

\newpage
\thispagestyle{empty}
\selectlanguage{english}
\addcontentsline{toc}{chapter}{Abstract}
\begin{abstract}

Higher education must adapt to the new constraints that we observe today such as the inability to attend
to attend face-to-face classes due to factors such as pandemics. Real-time communication technologies offer a good alternative
technologies offer a good alternative that is worth exploiting. In order to promote the use of these communication tools
we propose StudX, a prototype application for virtual classes. The development process of this 
The development process of this platform required the use of modern software modeling techniques, state-of-the-art technologies and appropriate resources.

\paragraph{}
\textbf{Key words}: StudX,WebRTC,online classes,universities,real time communication
\end{abstract}
\newpage

%liste des figures
\listoffigures 
\newpage

%liste des tableaux
\listoftables
\newpage

%liste des algo
\selectlanguage{french}
\listofalgorithmes
\newpage

% Les sigles et acronymes
\setglossarystyle{altlist}
\printglossary[title=Liste des acronymes, toctitle=Liste des acronymes, type=\acronymtype]
\newpage

% Le glossaire proprement dit
%\setglossarystyle{super}
%\printglossary[type=main]


\pagenumbering{arabic}
\setcounter{page}{1}
%%introduction

\introduction

\thispagestyle{plain} % removes unneeded heading

% TODO: Add a small corpus before the texts below

\section*{Contexte}
Les temps changent et les méthodes d’enseignement également. 
Dans le cadre de la mondialisation, on assiste à l'emploi du numérique, 
avec des méthodes de plus en plus créatives et collaboratives, 
pour une meilleure éducation. Cela reste valable dans le milieu de l’enseignement supérieur.
Il va sans dire que des facteurs comme la pandémie de COVID-19 ou encore, 
l'indisponibilité de cadres de cours adéquats, 
rendent plus urgent le besoin de transitionner 
vers des salles de classe virtuelle, pour répondre aux besoins. 
De ce fait, les technologies de l’information et 
de la communication constituent un atout décisif dans le succès de cette transition.

\section*{Problématique}
L’expansion des cours en ligne est désormais un fait. 
Cela requiert une organisation logistique accrue et un investissement financier pour les entités universitaires. 
Toutefois, l’on note l’emploi de solutions génériques, qui rendent difficile, 
voire impossible l'émulation d’un environnement de classe. 
Pour peu qu’elles soient conformes aux exigences, c’est alors le prix qui peut poser problème. 
C’est la raison d'être de notre projet, qui vise la mise en place d’une application, 
pour répondre au besoin d'interactivité lors des cours en ligne, 
et éliminer les barrières d’ordre logistique et financier, imposées par les solutions génériques.

\section*{Objectifs}
Le principal objectif est la conception de StudX, un prototype d’application SaaS, permettant la tenue de cours en ligne.
En termes de fonctionnalités et buts, il s’agira notamment de pouvoir :
\begin{itemize}
  \item organiser les différentes classes, filières ou promotions des entités en sections bien définies ;
  \item définir le calendrier des cours à tenir ;
  \item organiser des sessions d’audio-conférence pour le déroulement des cours ;
  \item mettre en place des fonctionnalités telles le partage d'écran et bien d’autre pour émuler un tant soit peu, 
    un environnement de classe présentiel ;
  \item minimiser les coûts requis dans le cadre de la mise en oeuvre d’une solution de classe virtuelle.
\end{itemize}

\thispagestyle{plain} % removes unneeded heading

\section*{Organisation du document}
Le présent document renferme trois chapitres. 
Dans le premier chapitre, nous ferons une revue de littérature sur le sujet 
et présenterons les généralités sur quelques notions essentielles. 
Le second chapitre relate les méthodes employées pour la conception de notre solution ainsi 
que les outils et matériels utilisés à cette fin. 
Le dernier chapitre sera consacré à la présentation des résultats obtenus, 
des interfaces conçues ainsi que des potentielles insuffisances liées à la solution que nous avons développée.

%\lhead[]{} \rhead[]{} \chead[]{}
\selectlanguage{french}
\fancyhead[L]{\tiny \leftmark}
\fancyhead[R]{\scriptsize \rightmark}
\fancyfoot[C]{\thepage}

\chapter{-}\label{chap:1}
 \addcontentsline{toc}{section}{Introduction}
\section*{Introduction}

\section{-}
blablabla

\section{-}
blablabla

\addcontentsline{toc}{section}{Conclusion}
\section*{Conclusion}

 
 \chapter{-}\label{chap:2}
 \addcontentsline{toc}{section}{Introduction}
\section*{Introduction}

\section{-}
blablabla

\section{-}
blablabla

\addcontentsline{toc}{section}{Conclusion}
\section*{Conclusion}

 
\chapter{--}\label{chap:3}
 \addcontentsline{toc}{section}{Introduction}
\section*{Introduction}

\section{-}
blablabla

\section{-}
\begin{algorithm}
	\KwData{$x$}
	\KwResult{$r$}
	\Begin{
		\If{$x \neq 0$}{
			$ r \leftarrow 1/x$\;
		}
	}
	
	\caption{Inverse}\label{alg:Inverse}
\end{algorithm}

\addcontentsline{toc}{section}{Conclusion}
\section*{Conclusion}
 
% \include{perspectives}
%%conclusion
\conclusion
Bla bla bla \cite{ehrig2006graph}
% 
\lhead[]{} \rhead[]{} \chead[]{}

%%biblio
\addcontentsline{toc}{chapter}{Bibliographie}
\bibliographystyle{abbrv}
\bibliography{biblio}


%\chapter*{Annexe}\addcontentsline{toc}{chapter}{Annexe}\label{annexe1}

\subsection*{Étapes clés du déroulement de l'attaque}


Nous allons exploiter quelques failles de ce réseau pour effectuer une attaque man in the middle (MITM).\\

Au début, notre machine Windows peut atteindre normalement le routeur R4.
\begin{figure}[H]
    \centering
    \includegraphics[scale=0.8]{images/ping_b4_1}
    \caption{Ping vers le routeur R4 avec succès}
    \label{fig:ping_b4_1}
\end{figure}
Quand on essaie de tracer le chemin vers R4, on constate que la machine passe par le routeur R1 légitime du lien pour atteindre R4
\begin{figure}[H]
    \centering
    \includegraphics{images/tracert_b4_1}
    \caption{Traces du chemin vers R4}
    \label{fig:tracert_b41}
\end{figure}
L'attaquant sur le lien peut alors passer a l'attaque.
Pour effectuer l'attaque MITM on utilisera l'outil fake\_router6, un utilitaire du package d'outils \textbf{the hacker choice}.
Ainsi sur la machine d'attaque, on active en un premier lieu le forwarding pour être transparent et ne pas bloquer le transit des paquets.
\begin{figure}[H]
    \centering
    \includegraphics{images/attk/fwrd_activation}
    \caption{Activation du forwarding des paquets.}
    \label{fig:activ_fwrd}
\end{figure}
Aussi on lance wireshark pour observer le trafic des paquets sur notre interface dans le réseau.\\
-------\\
Puisque tout est prêt nous allons lancer l'attaque.

\begin{figure}[H]
    \centering
    \includegraphics[scale=0.8]{images/attk/lancement_attk_1}
    \caption{Initialisation de l'attaque}
    \label{fig:attk_init_1}
\end{figure}

L'attaque est en cours et l'attaquant s'annonce comme le routeur par défaut du lien
nous allons maintenant vérifier la table des routes de notre machine windows.
\begin{figure}[H]
    \centering
    \includegraphics{images/attk/tableRoutes_windows}
    \caption{Table des routes de la machine victime}
    \label{fig:win_route_table}
\end{figure}
On constate que l'attaquant s'est insère comme passerelle de la victime.
pour confirmer cela reprenons un tracert vers le routeur r4
\begin{figure}[H]
    \centering
    \includegraphics{images/attk/tracert_b4_2}
    \caption{Chemin vers b4 pendant l'attaque.}
    \label{fig:tracert_b42}
\end{figure}
On peut voir clairement que la victime passe par l'attaquant pour atteindre le routeur.\\

A présent nous allons essayer de capturer une information envoyée par la victime.
Pour cela la victime fait un telnet sur le router R4 pour s'y connecter avec les paramètres suivants:\\
password1:\textbf{cisco}\\
password2:\textbf{class}
\begin{figure}[H]
    \centering
    \includegraphics{images/attk/telnet_r4}
    \caption{Connexion telnet au routeur.}
    \label{fig:telnetr4}
\end{figure}

Une fois la connexion réussie, nous allons voir avec wireshark les paquets de connexion et y retrouver les paramètres de connexion.
\begin{figure}[H]
    \centering
    \includegraphics[width=1.0\textwidth]{images/attk/c}
    \includegraphics[width=1.0\textwidth]{images/attk/i}
    \includegraphics[width=1.0\textwidth]{images/attk/s}
    \includegraphics[width=1.0\textwidth]{images/attk/c2}
    \includegraphics[width=1.0\textwidth]{images/attk/o}   
    \caption{Premier paramètre de connexion au routeur R4: \textbf{c-i-s-c-o}}
    \label{fig:param_conn_r4}
\end{figure}
\begin{figure}[H]
    \centering
    \includegraphics[width=1.0\textwidth]{images/attk/param2_c}
    \includegraphics[width=1.0\textwidth]{images/attk/param2_l}
    \includegraphics[width=1.0\textwidth]{images/attk/param2_a}
    \includegraphics[width=1.0\textwidth]{images/attk/param2_s1}
    \includegraphics[width=1.0\textwidth]{images/attk/param2_s2}
    \caption{Second paramètre de connexion au routeur R4: \textbf{c-l-a-s-s}}
    \label{fig:param_conn2}
\end{figure}
Les paramètres on été retrouves donc l'attaque a été un succès!

%\subsection*{Mitigations}
%Pour sécuriser ce réseau afin d'éviter ce genre d'attaque, deux mesures de sécurité peuvent être configurées.
%\begin{itemize}
%    \item le SEND
%    \item le RaGuard
%\end{itemize}



\newpage
\tableofcontents

\end{document}          
