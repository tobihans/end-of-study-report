%--------------A VOTRE ATTENTION-------------%
% Les étudiants en master qui disposent de plus de 3 chapitres dans leurs travaux peuvent en complèter
% Les Membres doivent figurer dans la dernière version finale du mémoire après soutenance pour dépôt de mémoire

\documentclass{ifri}
\usepackage{titletoc}
\setlength{\glsdescwidth}{0.65\textwidth}
% \usepackage{lscape}

\typeMemoire{Diplôme de Licence en Informatique}
\optionFormation{Sécurité Informatique}
\etudiant{Prenom \textbf{Nom}}
\titreDuMemoire{--} %Implémention pour une meilleure sécurité dans les réseau LAN sous IPv6 // Proposition: Identification des vulnérabilités dans un reseau LAN IPv6 et mesures pour une meilleure sécurité.

\dateSoutenance{-}
%\promo{2\up{ème}}
\anneeScolaire{--}


%%maitre de mémoire
\encadrants{Prenom \textbf{Nom}}

%% Membres du Jury
\jurys{%
\begin{tabular}{llll}
	Nom et prénoms du président & Grade & Entité & Président \\
	Nom et prénoms de l'examinateur & Grade & Entité & Examinateur \\
	Nom et prénoms du rapporteur & Grade & Entité & Rapporteur \\
\end{tabular}	
}


\hypersetup{
 pdftitle={--},
 pdfauthor={--},
 pdfsubject={--},
 pdfkeywords={--} 
 }

\color{bookColor}

%importation du glossaire
\loadglsentries{glossaire_reduit}

\begin{document}

\pageDeGarde
%\pageTitre

\pagecolor{white}

%% page vide
%\thispagestyle{empty}\ \clearpage


\selectlanguage{french}

% sommaire
\pagenumbering{roman}

\setcounter{tocdepth}{0}
\startlist{toc}
\printlist{toc}{}{\chapter*{Sommaire}}
\setcounter{tocdepth}{5}

%% rdedicaces
\dedicace
A \textit{Ma famille}, pour le soutien dont vous avez toujours fait montre.\newline

\newpage 

%% remerciements
\remerciements

Je tiens à exprimer mes sincères remerciements à toutes les personnes qui ont contribué de près ou de loin à la réalisation de ce mémoire. 

Tout d'abord, je remercie Mr. Eugène EZIN, directeur de l'institut, ainsi que tous les cadres de l'institut, pour m'avoir offert un environnement propice à l'apprentissage et au développement de mes compétences. 

Je suis également reconnaissant envers mes parents, pour leur soutien indéfectible tout au long de mon parcours universitaire.

Je souhaite exprimer ma profonde gratitude à mon maître de mémoire, Mme. Miranda GNONLONFOUN, pour son encadrement, ses conseils éclairés et son soutien constant durant tout le processus de rédaction de ce mémoire. 
Ses remarques et suggestions ont été très précieuses et ont contribué à améliorer significativement la qualité de ce travail.

Enfin, je tiens à remercier M. Chukwudi Nwachukwu, pour m'avoir enseigné bien des choses, sans lesquelles ce projet n'aurait su aboutir.

Encore une fois, je tiens à exprimer ma reconnaissance à toutes ces personnes qui, de près ou de loin, ont contribué à la réalisation de ce mémoire. Leur soutien, leurs encouragements et leurs conseils ont été inestimables et ont permis de faire de ce travail une réussite.
\newpage 

% Résume
\resume
\selectlanguage{french}
\vspace*{-6cm}
\begin{abstract}
L'enseignement supérieur est confronté à de nouvelles contraintes ; en témoigne la pandémie de COVID-19 qui a pertubé le déroulement
de cours présentiel. Les technologies de communication en temps réel offrent une alternative prometteuse, qu'il convient d'exploiter pour répondre à ces défis. 
Afin de favoriser l'utilisation de ces outils de communication, notre projet propose StudX, un prototype d'application permettant la tenue de classes virtuelles. 
Le développement de cette plateforme a nécessité l'utilisation de techniques modernes de modélisation logicielle, de technologies de pointe et de ressources variées. 
StudX offre une solution fiable pour répondre aux besoins des universités en matière d'enseignement à distance, grâce à des fonctionnalités de communication en temps réel basées sur WebRTC. 
Ce projet est une contribution importante à l'exploitation des technologies de communication en temps réel dans le domaine de l'éducation et offre des perspectives d'extension pour répondre à de nouveaux besoins.\newline\newline
\textbf{Mots clés}: StudX, WebRTC, cours en ligne, universités, communication en temps réel
\end{abstract}

\newpage
\thispagestyle{empty}
\selectlanguage{english}
\addcontentsline{toc}{chapter}{Abstract}
\begin{abstract}

Higher education is facing new constraints, such as the inability to attend face-to-face classes due to various factors such as pandemics. 
Real-time communication technologies offer a promising alternative that must be exploited to meet these challenges. 
In order to promote the use of these communication tools, our project proposes StudX, a prototype application for virtual classes. 
The development of this platform required the use of modern software modeling techniques, state-of-the-art technologies and various resources. 
StudX provides a reliable solution to the distance learning needs of universities through WebRTC-based real-time communication capabilities. 
This project is an important contribution to the exploitation of real-time communication technologies in the field of education and has the potential to be extended to meet new needs.\newline\newline
\textbf{Keywords}: StudX, WebRTC, online classes, universities, real time communication
\end{abstract}
\newpage

%liste des figures
\listoffigures 
\newpage

%liste des tableaux
\listoftables
\newpage

%liste des algo
\selectlanguage{french}
\listofalgorithmes
\newpage

% Les sigles et acronymes
\setglossarystyle{altlist}
\printglossary[title=Liste des acronymes, toctitle=Liste des acronymes, type=\acronymtype]
\newpage

% Le glossaire proprement dit
%\setglossarystyle{super}
%\printglossary[type=main]


\pagenumbering{arabic}
\setcounter{page}{1}
%%introduction
\introduction
bla bla bla \gls{acro} puis \Gls{acroglo} et enfin \gls{glossaire}
%\lhead[]{} \rhead[]{} \chead[]{}
\selectlanguage{french}
\fancyhead[L]{\tiny \leftmark}
\fancyhead[R]{\scriptsize \rightmark}
\fancyfoot[C]{\thepage}

\chapter{-}\label{chap:1}
 \addcontentsline{toc}{section}{Introduction}
\section*{Introduction}

\section{-}
blablabla

\section{-}
blablabla

\addcontentsline{toc}{section}{Conclusion}
\section*{Conclusion}

 
 \chapter{-}\label{chap:2}
 \addcontentsline{toc}{section}{Introduction}
\section*{Introduction}

\section{-}
blablabla

\section{-}
blablabla

\addcontentsline{toc}{section}{Conclusion}
\section*{Conclusion}

 
\chapter{--}\label{chap:3}
 \addcontentsline{toc}{section}{Introduction}
\section*{Introduction}

\section{-}
blablabla

\section{-}
blablabla

\addcontentsline{toc}{section}{Conclusion}
\section*{Conclusion}

 
% \include{perspectives}
%%conclusion
\conclusion
L'évolution des méthodes d'enseignement implique l'utilisation croissante des technologies de l'information et de la communication. 
Cette tendance est motivée par la nécessité de s'adapter aux nouveaux besoins d'apprentissage et de compenser les insuffisances logistiques et 
financières auxquelles font face les établissements d'enseignement supérieur. Dans ce contexte, 
le projet auquel nous avons contribué visait à fournir un cadre moderne de communication en temps réel, 
permettant d'explorer de nouvelles possibilités en matière de formation virtuelle. 
Nous sommes convaincus que l'application que nous avons développée ouvre la voie à une large gamme d'opportunités en matière d'exploitation des classes virtuelles, 
et nous espérons que notre travail sera une contribution utile à l'ensemble de la communauté éducative.  
Bien qu'ayant atteint ses objectifs initiaux et répondu aux besoins identifiés, il reste encore de nombreuses possibilités d'extension pour en faire un outil encore plus puissant et adapté aux besoins évolutifs de l'enseignement en ligne. 
Il est donc envisageable d'explorer des pistes de développement pour étendre l'application au-delà de ses fonctions actuelles.

\section*{Perspectives}
Le prototype StudX propose une variété de fonctionnalités intéressantes pour la tenue de classes virtuelles.  
Toutefois, il présente certaines limites, outre les choix délibérés de conception comme l’absence de flux vidéo. 

Bien que des règles d'accessibilité basiques aient été prises en compte, elles ne répondent pas entièrement aux 
besoins des personnes malentendantes, qui ne peuvent pas bénéficier des échanges vocaux entre les participants. 
Pour remédier à cela, il serait envisageable d'intégrer un modèle de Machine Learning pour la conversion des signaux 
audio en gestes du langage des signes.

Par ailleurs, dans le cadre d’un prototype, les fonctionnalités sont plutôt restreintes. 
On pourrait ajouter ou améliorer des fonctions comme la persistance de la messagerie instantanée ou 
encore la gestion des utilisateurs. L'enregistrement des sessions pour un usage ultérieur, par exemple, est une fonctionnalité
dont l'implémentation a démarré (branche \href{https://github.com/tobihans/studx/tree/features/room-recording}{room-recording}) dans le but de permettre une amélioration de l'experience utilisateur.

\addcontentsline{toc}{section}{Conclusion}

% 
\lhead[]{} \rhead[]{} \chead[]{}

%%biblio
\addcontentsline{toc}{chapter}{Bibliographie}
\bibliographystyle{abbrv}
\bibliography{biblio}


%\chapter*{Annexe}\addcontentsline{toc}{chapter}{Annexe}\label{annexe1}

\subsection*{Étapes clés du déroulement de l'attaque}


Nous allons exploiter quelques failles de ce réseau pour effectuer une attaque man in the middle (MITM).\\

Au début, notre machine Windows peut atteindre normalement le routeur R4.
\begin{figure}[H]
    \centering
    \includegraphics[scale=0.8]{images/ping_b4_1}
    \caption{Ping vers le routeur R4 avec succès}
    \label{fig:ping_b4_1}
\end{figure}
Quand on essaie de tracer le chemin vers R4, on constate que la machine passe par le routeur R1 légitime du lien pour atteindre R4
\begin{figure}[H]
    \centering
    \includegraphics{images/tracert_b4_1}
    \caption{Traces du chemin vers R4}
    \label{fig:tracert_b41}
\end{figure}
L'attaquant sur le lien peut alors passer a l'attaque.
Pour effectuer l'attaque MITM on utilisera l'outil fake\_router6, un utilitaire du package d'outils \textbf{the hacker choice}.
Ainsi sur la machine d'attaque, on active en un premier lieu le forwarding pour être transparent et ne pas bloquer le transit des paquets.
\begin{figure}[H]
    \centering
    \includegraphics{images/attk/fwrd_activation}
    \caption{Activation du forwarding des paquets.}
    \label{fig:activ_fwrd}
\end{figure}
Aussi on lance wireshark pour observer le trafic des paquets sur notre interface dans le réseau.\\
-------\\
Puisque tout est prêt nous allons lancer l'attaque.

\begin{figure}[H]
    \centering
    \includegraphics[scale=0.8]{images/attk/lancement_attk_1}
    \caption{Initialisation de l'attaque}
    \label{fig:attk_init_1}
\end{figure}

L'attaque est en cours et l'attaquant s'annonce comme le routeur par défaut du lien
nous allons maintenant vérifier la table des routes de notre machine windows.
\begin{figure}[H]
    \centering
    \includegraphics{images/attk/tableRoutes_windows}
    \caption{Table des routes de la machine victime}
    \label{fig:win_route_table}
\end{figure}
On constate que l'attaquant s'est insère comme passerelle de la victime.
pour confirmer cela reprenons un tracert vers le routeur r4
\begin{figure}[H]
    \centering
    \includegraphics{images/attk/tracert_b4_2}
    \caption{Chemin vers b4 pendant l'attaque.}
    \label{fig:tracert_b42}
\end{figure}
On peut voir clairement que la victime passe par l'attaquant pour atteindre le routeur.\\

A présent nous allons essayer de capturer une information envoyée par la victime.
Pour cela la victime fait un telnet sur le router R4 pour s'y connecter avec les paramètres suivants:\\
password1:\textbf{cisco}\\
password2:\textbf{class}
\begin{figure}[H]
    \centering
    \includegraphics{images/attk/telnet_r4}
    \caption{Connexion telnet au routeur.}
    \label{fig:telnetr4}
\end{figure}

Une fois la connexion réussie, nous allons voir avec wireshark les paquets de connexion et y retrouver les paramètres de connexion.
\begin{figure}[H]
    \centering
    \includegraphics[width=1.0\textwidth]{images/attk/c}
    \includegraphics[width=1.0\textwidth]{images/attk/i}
    \includegraphics[width=1.0\textwidth]{images/attk/s}
    \includegraphics[width=1.0\textwidth]{images/attk/c2}
    \includegraphics[width=1.0\textwidth]{images/attk/o}   
    \caption{Premier paramètre de connexion au routeur R4: \textbf{c-i-s-c-o}}
    \label{fig:param_conn_r4}
\end{figure}
\begin{figure}[H]
    \centering
    \includegraphics[width=1.0\textwidth]{images/attk/param2_c}
    \includegraphics[width=1.0\textwidth]{images/attk/param2_l}
    \includegraphics[width=1.0\textwidth]{images/attk/param2_a}
    \includegraphics[width=1.0\textwidth]{images/attk/param2_s1}
    \includegraphics[width=1.0\textwidth]{images/attk/param2_s2}
    \caption{Second paramètre de connexion au routeur R4: \textbf{c-l-a-s-s}}
    \label{fig:param_conn2}
\end{figure}
Les paramètres on été retrouves donc l'attaque a été un succès!

%\subsection*{Mitigations}
%Pour sécuriser ce réseau afin d'éviter ce genre d'attaque, deux mesures de sécurité peuvent être configurées.
%\begin{itemize}
%    \item le SEND
%    \item le RaGuard
%\end{itemize}



\newpage
\tableofcontents

\end{document}          
