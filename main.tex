\documentclass{ifri}
\usepackage{titletoc}
\usepackage{fancyhdr}
\graphicspath{ {./images/} }
\setlength{\glsdescwidth}{0.65\textwidth}
% \usepackage{lscape}

\typeMemoire{Diplôme de Licence en Informatique}
\optionFormation{Génie Logiciel}
\etudiant{Hans Bignon. K. \textbf{TOGNON}}
\titreDuMemoire{Conception d'un SaaS pour les cours en ligne en milieu universitaire}

\dateSoutenance{-}
%\promo{2\up{ème}}
\anneeScolaire{2021-2022}


%%maitre de mémoire
\encadrants{Miranda \textbf{GNONLONFOUN}}

%% Membres du Jury
\jurys{%
\begin{tabular}{llll}
	Nom et prénoms du président & Grade & Entité & Président \\
	Nom et prénoms de l'examinateur & Grade & Entité & Examinateur \\
	Nom et prénoms du rapporteur & Grade & Entité & Rapporteur \\
\end{tabular}	
}


\hypersetup{
 pdftitle={Conception d'un SaaS pour les cours en ligne en milieu universitaire},
 pdfauthor={Hans TOGNON},
 pdfsubject={--},
 pdfkeywords={Sass, Studx, WebRTC, online, classes}
 }

\color{bookColor}

%importation du glossaire
\loadglsentries{glossaire}

\begin{document}

\pageDeGarde
%\pageTitre

\pagecolor{white}

%% page vide
%\thispagestyle{empty}\ \clearpage


\selectlanguage{french}

% sommaire
\pagenumbering{roman}

\setcounter{tocdepth}{0}
\startlist{toc}
\printlist{toc}{}{\chapter*{Sommaire}}
\setcounter{tocdepth}{5}

%% rdedicaces
\dedicace
A \textit{Ma famille}, pour le soutien dont vous avez toujours fait montre.\newline

\newpage 

%% remerciements
\remerciements

Je tiens à exprimer mes sincères remerciements à toutes les personnes qui ont contribué de près ou de loin à la réalisation de ce mémoire. 

Tout d'abord, je remercie Mr. Eugène EZIN, directeur de l'institut, ainsi que tous les cadres de l'institut, pour m'avoir offert un environnement propice à l'apprentissage et au développement de mes compétences. 

Je suis également reconnaissant envers mes parents, pour leur soutien indéfectible tout au long de mon parcours universitaire.

Je souhaite exprimer ma profonde gratitude à mon maître de mémoire, Mme. Miranda GNONLONFOUN, pour son encadrement, ses conseils éclairés et son soutien constant durant tout le processus de rédaction de ce mémoire. 
Ses remarques et suggestions ont été très précieuses et ont contribué à améliorer significativement la qualité de ce travail.

Enfin, je tiens à remercier M. Chukwudi Nwachukwu, pour m'avoir enseigné bien des choses, sans lesquelles ce projet n'aurait su aboutir.

Encore une fois, je tiens à exprimer ma reconnaissance à toutes ces personnes qui, de près ou de loin, ont contribué à la réalisation de ce mémoire. Leur soutien, leurs encouragements et leurs conseils ont été inestimables et ont permis de faire de ce travail une réussite.
\newpage 

% Résume
\resume
\selectlanguage{french}
\vspace*{-6cm}
\begin{abstract}
L'enseignement supérieur est confronté à de nouvelles contraintes ; en témoigne la pandémie de COVID-19 qui a pertubé le déroulement
de cours présentiel. Les technologies de communication en temps réel offrent une alternative prometteuse, qu'il convient d'exploiter pour répondre à ces défis. 
Afin de favoriser l'utilisation de ces outils de communication, notre projet propose StudX, un prototype d'application permettant la tenue de classes virtuelles. 
Le développement de cette plateforme a nécessité l'utilisation de techniques modernes de modélisation logicielle, de technologies de pointe et de ressources variées. 
StudX offre une solution fiable pour répondre aux besoins des universités en matière d'enseignement à distance, grâce à des fonctionnalités de communication en temps réel basées sur WebRTC. 
Ce projet est une contribution importante à l'exploitation des technologies de communication en temps réel dans le domaine de l'éducation et offre des perspectives d'extension pour répondre à de nouveaux besoins.\newline\newline
\textbf{Mots clés}: StudX, WebRTC, cours en ligne, universités, communication en temps réel
\end{abstract}

\newpage
\thispagestyle{empty}
\selectlanguage{english}
\addcontentsline{toc}{chapter}{Abstract}
\begin{abstract}

Higher education is facing new constraints, such as the inability to attend face-to-face classes due to various factors such as pandemics. 
Real-time communication technologies offer a promising alternative that must be exploited to meet these challenges. 
In order to promote the use of these communication tools, our project proposes StudX, a prototype application for virtual classes. 
The development of this platform required the use of modern software modeling techniques, state-of-the-art technologies and various resources. 
StudX provides a reliable solution to the distance learning needs of universities through WebRTC-based real-time communication capabilities. 
This project is an important contribution to the exploitation of real-time communication technologies in the field of education and has the potential to be extended to meet new needs.\newline\newline
\textbf{Keywords}: StudX, WebRTC, online classes, universities, real time communication
\end{abstract}
\newpage

%liste des figures
\listoffigures 
\newpage

%liste des tableaux
\listoftables
\newpage

%liste des algo
\selectlanguage{french}
\newpage

% Les sigles et acronymes
\setglossarystyle{altlist}
\printglossary[title=Liste des acronymes, toctitle=Liste des acronymes, type=\acronymtype]
\newpage

% Le glossaire proprement dit
%\setglossarystyle{super}
%\printglossary[type=main]


\pagenumbering{arabic}
\setcounter{page}{1}
%%introduction
\introduction
bla bla bla \gls{acro} puis \Gls{acroglo} et enfin \gls{glossaire}
%\lhead[]{} \rhead[]{} \chead[]{}
\selectlanguage{french}
\fancyhead[L]{\tiny \leftmark}
\fancyhead[R]{\scriptsize \rightmark}
\fancyfoot[C]{\thepage}

\chapter{Revue de littérature}\label{chap:1}
\addcontentsline{toc}{section}{Introduction}
\section*{Introduction}

Le concept de la formation à distance ne date pas d’hier. 
Dans ce chapitre, nous ferons une revue des origines de cette méthode d’enseignement. 
S’en suivra une analyse des techniques modernes de communication en temps réel et des solutions existantes qui permettent  de dispenser des cours à distance.

\section{Formation à distance}

L'encyclopédie Wikipedia définit la formation à distance comme une 
forme d’enseignement ou l’enseignant et l'étudiant sont séparés 
dans le temps et/ou par l’espace \cite{distance_education}.

\subsection{Origines}

Les premiers essais de formation à distance remontent à bien avant l'ère moderne. 
En effet, déjà en 1728, Caleb Phillips, un professeur, 
recherchaient des étudiants désirant acquérir des compétences en sténographie, 
auxquels il dispensait les cours par courrier.

Au sens moderne, le premier cours d’enseignement à distance, est attribué à Isaac Pitman, toujours en rapport avec la sténographie. 
Un nouvel élément qui apparaît dans le cas actuel, c’est la rétroaction des étudiants, 
qui devaient envoyer leurs transcriptions par la poste, pour correction. 
Ce mode de fonctionnement fut rendu possible par l’uniformisation des tarifs postaux en Angleterre. 
Plusieurs institutions telles que Oxford et l'université de Londre ont également expérimenté l’enseignement à distance.

Aujourd’hui, l’expansion d’Internet et du World Wide Web, permettent la mise en œuvre des moyens toujours plus sophistiqués, pour dispenser les cours à distance.


\subsection{Internet et formations en ligne}

L'avènement des nouvelles technologies de l’information et de la communication 
a donné lieu à la mise en place des formations en ligne, 
une forme évoluée de formation à distance \cite{online_learning}. 
En 2020, par exemple, sous l’impulsion de la pandémie alors en cours, 
plusieurs universités dont celle d’Abomey-Calavi, 
effectuent la transition partielle ou totale vers les classes virtuelles \cite{etudiant_dot_bj}.

On distingue deux environnements d’apprentissage. 
D’une part, les environnements asynchrones, qui offrent une totale liberté à l'apprenant quant à la gestion de son temps. 
L’enseignant et lui sont séparés littéralement par le temps et la distance. Ainsi, il peut consulter les ressources au moment qui lui convient le mieux. 
Cela permet une assimilation plus facile, étant donné que chaque apprenant peut s’adapter en fonction de ses besoins spécifiques. 
Toutefois, il est possible que l’apprenant se retrouve isolé et ne fasse finalement aucun progrès, faute de support.

D’autre part, un environnement synchrone essaie d'émuler une classe présentiel, 
à la seule différence que les participants sont physiquement distants. 
Avec des outils de messagerie instantanée et/ou de visioconférence, 
les apprenants peuvent interagir avec leurs pairs ainsi que le ou les enseignants.

En termes de classification des diverses formes de cours en ligne, Andreas Kaplan, 
auteur du livre \textit{Contemporary Issues in Social Media Marketing} \cite{andreas_kaplan_classfication}, propose une approche simplifiée basée sur les facteurs 
comme le temps, la distance et le nombre d’apprenants. 
Le tableau \ref{table:andreas_k_dot} en fait un récapitulatif.

\begin{table}[h!]
  \centering
\begin{tabular}{|l|l|l|}
  \hline
  Type de cours &  Nombre d’apprenants &  Type d’environnement \\
  \hline
  \acrfull{mooc} & illimité(en théorie) & asynchrone \\
  \hline
  \acrfull{smoc} & illimité(en théorie) & synchrone \\
  \hline
  \acrfull{spoc} & limité & asynchrone \\
  \hline
  \acrfull{ssoc} & limité & synchrone \\
  \hline
\end{tabular}
\caption{Classfication des types de formations en ligne}
\label{table:andreas_k_dot}
\end{table}

Notons que notre étude s'intéresse notamment aux environnements d’apprentissage synchrones, 
avec le support d’un grand nombre d’apprenants (\acrshort{smoc}s).

\section{Communication en temps réel}
Les outils de communications en temps réel désignent une catégorie de logiciels qui garantissent le traitement et la transmission instantanée, 
ou avec un délai fortement négligeable,  de l’information. 
Parmi les protocoles permettant ce type de communication, le plus en vogue reste \acrfull{webrtc}.


\acrshort{webrtc} est un protocole open source de transmission \gls{p2p}, qui assure la transmission de média (audio, vidéo) 
et de données brutes, presque sans latence (moins d’une seconde), 
le tout dans un contexte hautement sécurisé. 
Il s’agit en réalité, d’une collection de protocoles datant des années 2000. 
Pour établir une connexion, il faut quatre étapes à savoir la signalisation, la connexion proprement dite, la sécurisation puis la communication.

\newpage
\begin{figure}[h]
  \centering
  \frame{\includegraphics[width=0.75\textwidth]{webrtc-protocol-stack}}
  \caption{Protocoles employés par la technologie WebRTC}
  \label{fig:webrtc_protocols}
\end{figure}

La signalisation désigne le processus initial de mise en relation des pairs. 
Sans ce processus, une machine quelconque n’a aucune idée de qui voudrait bien la contacter. 
Pour ce faire, le protocole \acrshort{sdp} est utilisé et permet la transmission d’informations capitales comme:

\begin{itemize}
  \item l’adresse \acrshort{ip} et le port de chaque agent \acrshort{webrtc} (plusieurs variantes, en réalité)
  \item les codecs multimédia supportés
  \item d’autres valeurs comme des certificats de sécurité nécessaires à la mise en place de la connexion et la sécurisation.
\end{itemize}


A la connexion, les agents \acrshort{webrtc} établissent un lien direct entre eux, sans intermédiaire. 
Face à la multitude de possibilités de connexion (couples constitués de l’adresse \acrshort{ip} et du port), 
le protocole \acrshort{ice} permet de choisir le meilleur candidat, en faisant recours au serveur \acrshort{stun} et parfois, à un serveur \acrshort{turn}. 
Le serveur \acrshort{turn} permet la retransmission des données lorsqu’il est impossible pour un agent \acrshort{webrtc} d'établir un lien 
direct avec un autre agent en raison de la configuration réseau (\acrshort{nat} et les types de liaisons possibles \cite{nat_links}).

Pour assurer la sécurité de la connexion, les protocoles \acrshort{dtls} et \acrshort{srtp} offrent une couche de chiffrement 
pour les contenus multimédia et les paquets brutes.

Enfin, les agents peuvent s'échanger de la donnée, du contenu multimédia, presque sans latence, grâce au protocoles \acrshort{rtp} et \acrshort{sctp}.

\acrshort{webrtc} est une technologie complexe qui requiert une certaine expertise quant à la connaissance des protocoles, 
leur utilisation et la mise en œuvre d'applications en temps réel. 
Elle sert de base aujourd’hui,  la plupart des applications de communication en temps réel.

\section{Software as a Service}
Parmi les modèles de distribution de logiciels, le SaaS représente une méthode ou le concepteur ou l'entité tenant l’application, 
l'héberge en ligne et la rend accessible à ses utilisateurs. 
En terme de commercialisation, il est possible d’offrir un accès à la plateforme moyennant un abonnement ou 
l’achat d’une version privée pour les besoins des corporations.

\section{Présentation de solutions existantes}
Plusieurs solutions s’inscrivent déjà dans le cadre du déroulement de cours en ligne en temps réel. 
Nous avons choisi quelques unes à passer en revue.

Il est important de préciser que les insuffisances relevées par rapport à ces outils ne sont aucunement d’ordre technique. 
Nous nous intéressons plutôt aux aspects logistique et financier. 
En effet, un des objectifs visés est de minimiser l’investissement requis pour la mise en place d’une solution de classe en ligne, 
tout en éliminant les barrières possibles.

\subsection{Google Classroom}
Google Classroom est un outil de la suite Google pour l'éducation. 
A défaut de disposer d’un module de visioconférence, il s'intègre parfaitement avec Google Meet (figure \ref{fig:g_meet}), à cette fin. 
L’application offre une version gratuite et dispose d’une interface accessible (figure \ref{fig:g_classrooms}). 
Toutefois, pour les réunions en ligne, le nombre maximum de connexions possibles se limite à 500 participants. 
Pour les entités universitaires dont l’effectif est considérable par classe, ceci pourrait présenter un désavantage. 
Grâce à la version payante néanmoins, on peut mettre en place un live stream, pour permettre d'accéder au contenu de 
la réunion sans toutefois pouvoir interagir avec les participants. 
Mais le modèle de souscription (figure \ref{fig:g_pricing}), basé sur le nombre d’utilisateurs risque d'entraîner des frais assez élevés.

% illustrations
\begin{figure}[h]
  \centering
  \frame{\includegraphics[width=0.75\textwidth]{g-classrooms}}
  \caption{Page d'acceuil de Google Classrooms avec un cas réel de cours}
  \label{fig:g_classrooms}
\end{figure}

\break

\begin{figure}[h]
  \centering
  \frame{\includegraphics[width=0.75\textwidth]{g-meet}}
  \caption{Utilisation de Google Meet}
  \label{fig:g_meet}
\end{figure}
\begin{figure}[h]
  \centering
  \frame{\includegraphics[width=0.75\textwidth]{gsuite-pricing-2023-02-17}}
  \caption{Offres de souscription à la suite Google pour éducation}
  \label{fig:g_pricing}
\end{figure}

\subsection{Zoom}
Zoom est un outil de communication très performant, qui a la capacité de supporter un grand nombre d’utilisateurs. 
Il dispose de fonctionnalités très utiles pour le déroulement de cours en ligne comme le partage d'écran ou le tableau virtuel. 
Accéder à ces fonctionnalités dans le cadre d’une utilisation à grande échelle requiert une souscription et les offres de Zoom ne sont pas des plus simples. 
En effet, Zoom dispose d’un panel large de services associés (figure \ref{fig:zoom_pricing}) et donc, sans orientation, 
il est possible de choisir une solution inadéquate en rapport avec le besoin, sans compter la perte financière.

% illustrations
\newpage
\begin{figure}[h]
  \centering
  \includegraphics[width=0.75\textwidth]{zoom-meeting-ui}
  \caption{Interface de l'application Zoom}
  \label{fig:zoom_meeting_ui}
\end{figure}

\begin{figure}[h]
  \centering
  \frame{\includegraphics[width=0.75\textwidth]{zoom-pricing-2023-02-17}}
  \caption{Interface de l'application Zoom}
  \label{fig:zoom_pricing}
\end{figure}

\subsection{Moodle}
Moodle est un LMS Open Source populaire très connu et utilisé dans les entités de l’enseignement supérieur. 
Il peut être hébergé ou utilisé en ligne. 
Il offre un large panel de fonctionnalités et permet l'intégration de divers modules dont des modules de visio-conférence. 
BigBlueButton (figure \ref{fig:bbg_demo}) est une solution Open source employée à cet effet. 
La mise en place requiert toutefois, une certaine expertise et du matériel spécifique, ce qui en limiterait la portabilité.


% illustrations
\newpage
\begin{figure}[h]
  \centering
  \frame{\includegraphics[width=0.75\textwidth]{moodle-demo-site}}
  \caption{Démonstration des capacités de Moodle}
  \label{fig:moodle_demo}
\end{figure}

\begin{figure}[h]
  \centering
  \frame{\includegraphics[width=0.75\textwidth]{big-blue-demo}}
  \caption{Démonstration de BigBlueButton}
  \label{fig:bbg_demo}
\end{figure}

\addcontentsline{toc}{section}{Conclusion}
\section*{Conclusion}
Ce chapitre a permis de faire une revue de l’existant et jette les bases des 
suivants en exposant les concepts clés qui seront développés. 
Les solutions suscitées conviendraient pour un usage modéré. 
Elles peuvent s'avérer coûteuses, pour peu qu’elles répondent au besoin. 
La solution que nous proposons vise à doter les organismes de l’enseignement supérieur, 
d’un moyen simple mais efficace de tenir les cours en ligne, offrant des outils d’assistance, 
tout en minimisant les coûts, que cela pourrait engendrer.


\chapter{Matériels et méthodes}\label{chap:2}
\addcontentsline{toc}{section}{Introduction}
\section*{Introduction}
Ce chapitre est dédié à la mise en lumière des pratiques d’architecture logicielle 
employées lors de la conception de notre prototype. 
Nous y présenterons également les choix techniques effectués.

\section{Méthodes de conception}
Dans le souci de décrire de façon fiable, les fonctionnalités du système, nous faisons usage du langage visuel \acrfull{uml}. 
Il s’agit d’une méthode de visualisation d’architecture logicielle permettant de modéliser 
l’architecture logicielle d’un système.

Standardisé par \acrshort{omg}, la version actuelle de \acrshort{uml}, la 2.5 \cite{uml_spec_link}, propose 14 types de diagrammes. 
N'étant pas une méthode, la norme laisse l’utilisation des diagrammes à l'appréciation des utilisateurs.
Dans le cadre de notre prototype, nous avons retenu uniquement les diagrammes de cas d’utilisation, 
de séquence et de classe, car ils expriment bien la structure de notre application.

\subsection{Diagramme de cas d’utilisation}
Les diagrammes de cas d’utilisation illustrent le comportement fonctionnel du système. 
Les cas d’utilisation sont utiles pour décrire les interactions possibles entre 
le/les acteurs acteur(s) et le système.
 
Les acteurs intervenant dans notre système sont:

\begin{itemize}
  \item L’étudiant : il dispose d’un accès en lecture aux informations du système ;
  \item L’enseignant : il dispose d’un accès total en lecture et partiel en écriture sur certaines informations ;
  \item L’administrateur : il dispose de privilèges élevés pour modifier les informations de la plateforme et l’administrer.
\end{itemize}

La figure \ref{fig:use_case_diag} en fait l'illustration.

\begin{figure}[H]
  \centering
  \includegraphics[width=\linewidth]{use-cases-diag}
  \caption{Diagramme de cas d'utilisation du prototype StudX}
  \label{fig:use_case_diag}
\end{figure}


Outre le diagramme, il s'avère parfois nécessaire de fournir, en plus, des descriptions textuelles des cas 
d’utilisation, dans le but d’apporter plus d'éclaircissements. 
Ci-dessous, sont présentées les descriptions textuelles des cas “Ajouter un événement” et “Rejoindre une réunion”.

\subsubsection{Description textuelle du cas d’utilisation “Ajouter un événement”}
\textbf{Titre} : Ajouter un événement\newline
\textbf{Objectif} : Planifier les cours en ligne en ajoutant des événements au calendrier\newline
\textbf{Acteurs} : Enseignant ou Administrateur\newline
\textbf{Pré-conditions} : 
\begin{itemize}[noitemsep,topsep=0pt]
  \item l’utilisateur est authentifié en tant qu’enseignant ou administrateur.
\end{itemize}
\textbf{Séquence nominale} :
\begin{enumerate}[noitemsep,topsep=0pt]
  \item L’utilisateur accède au calendrier ;
	\item Le système renvoie les événements actuellement programmes ;
	\item L’utilisateur remplit et soumet un formulaire de création ;
	\item Le système enregistre l'événement et les détails associés ;
	\item Le système notifie les participants concernés.
\end{enumerate}
\textbf{Post-conditions} : 
\begin{itemize}[noitemsep,topsep=0pt]
  \item L’utilisateur accède à l'événement dans son calendrier.
\end{itemize}

\subsubsection{Description textuelle du cas d’utilisation “Rejoindre une réunion”}
\textbf{Titre} : Rejoindre une réunion\newline
\textbf{Objectif} : Tenir une session de classe en ligne\newline
\textbf{Acteurs} : Étudiant ou Enseignant ou Administrateur\newline
\textbf{Pré-conditions} : 
\begin{itemize}[noitemsep,topsep=0pt]
  \item l’utilisateur est authentifié.
\end{itemize}
\textbf{Séquence nominale} :
\begin{enumerate}[noitemsep,topsep=0pt]
  \item  l’utilisateur consulte le calendrier ;
	\item le système affiche les divers événements programmes ;
	\item l’utilisateur accède aux détails d’un événement ;
	\item l’utilisateur clique sur le lien pour rejoindre la reunion ;
	\item le système connecté l’utilisateur aux participants présents.
\end{enumerate}
\textbf{Post-conditions}: 
\begin{itemize}[noitemsep,topsep=0pt]
  \item L’utilisateur est en mesure d’interagir, de communiquer avec les participants.
\end{itemize}

\subsection{Diagramme de séquence}
Le diagramme de séquence décrit les interactions, dans l’espace temps, entre objets dans le cadre des scénarii évoqués au niveau des cas d’utilisations. 
Les figures \ref{fig:add_event_seq_diag} et \ref{fig:join_meet_seq_diag} illustrent les diagrammes de séquences pour les deux cas d’utilisation suscités.


\begin{figure}[H]
  \centering
  \includegraphics[width=\linewidth]{add-event-sequence-diag}
  \caption{Diagramme de séquence pour le cas d’utilisation “Ajouter un événement”}
  \label{fig:add_event_seq_diag}
\end{figure}

\begin{figure}[H]
  \centering
  \includegraphics[width=\linewidth]{join-meet-sequence-diag}
  \caption{Diagramme de séquence pour le cas d’utilisation “Rejoindre une réunion”}
  \label{fig:join_meet_seq_diag}
\end{figure}

\subsection{Diagramme de classe}
Le diagramme de classe illustre les classes et les interfaces du système ainsi que les relations qui les lient. 
Le diagramme à la figure \ref{fig:class_diag} en dessous décrit les diverses entités de notre prototype.

\begin{figure}[H]
  \centering
  \includegraphics[width=\linewidth]{class-diag}
  \caption{Diagramme de classe}
  \label{fig:class_diag}
\end{figure}

\subsection{Architecture du système}
Pour assurer la scalabilité des systèmes, il est important de bien en concevoir l’architecture. 
Dans ce but, nous avons adopté une approche découplée, isolant les composantes du système. 
Il s’agit de microservices. Toutefois, il est important de noter qu'à l'échelle d’un prototype, 
l’architecture proposée reste très simplifiée et ne prend pas en compte des facteurs comme la résilience \cite{microservices_resiliency}. 
La figure de l'encart \ref{fig:system_design} présente les diverses composantes de notre architecture.


\begin{figure}[H]
  \centering
  \frame{\includegraphics[width=\linewidth]{studx-system-design}}
  \caption{Architecture du système du prototype StudX}
  \label{fig:system_design}
\end{figure}

On peut notamment remarquer que toutes les interactions entre le système et les utilisateurs passent toutes par un proxy. 
Ceci s’explique principalement par la volonté d'éviter les problèmes de CORS, 
qui occurrent dès lors que les services ne sont pas tous sur un même domaine.

Outre ces détails, il faut préciser également que le prototype implémente l’architecture de multi-entité \cite{multitenancy}, 
en regroupant les utilisateurs et les données qui leur sont communes en entités que nous qualifions d’organisation. 
Le but est de pouvoir servir plusieurs regroupements sans pour autant avoir à répliquer le matériel. 

\section{Matériels}
\subsection{Choix techniques}
Faisant référence à l’architecture suscitée, 
voyons à présent les technologies employées dans la mise en place de la solution.

\subsubsection{Proxy}
Un serveur proxy sert de relais entre différentes parties, notamment entre le client et le serveur dans notre contexte. 
Il s’agit dans ce cas, d’un reverse proxy, car le relai va du client vers le serveur.
 
\textbf{Caddy} est un serveur Web moderne qui offre un large panel de fonctionnalités. 
Il offre un large éventail de fonctionnalités que l’on peut mettre en place via un fichier de configuration spéciale nommé Caddyfile. 
La figure \ref{fig:caddyfile} en présente un exemple, extrait du code de notre prototype.


\begin{figure}[H]
  \centering
  \includegraphics[width=\linewidth]{caddyfile}
  \caption{Example de configuration du proxy \textbf{Caddy}}
  \label{fig:caddyfile}
\end{figure}

On peut remarquer entre autres, l’utilisation de variables d’environnement qui permettent de rendre la configuration encore plus dynamique. 
Tous ces atouts en font un bon choix pour notre prototype.

\subsubsection{API REST}

Une \acrshort{api} \acrshort{rest} définit interface de programmation respectant les contraintes du style d’architecture \acrlong{rest}.

Elle doit disposer des caractéristiques suivantes :

\begin{itemize}
  \item Une architecture client-serveur constituée de clients, de serveurs et de ressources, avec des requêtes gérées via \acrshort{http} ;
  \item Des communications client-serveur sans etat, c'est-à-dire que les informations du client ne sont jamais stockées entre les requêtes, 
  qui doivent être traitées séparément, de manière totalement indépendante ;
  \item La possibilité de mettre en cache des données afin de rationaliser les interactions client-serveur ;
  \item Une interface uniforme entre les composants qui permet un transfert standardisé des informations ;
  \item Un système à couches, invisible pour le client, qui permet de hiérarchiser les différents types de serveurs 
  (pour la sécurité, l'équilibrage de charge, etc.) impliqués dans la récupération des informations demandées ;
  \item Du code à la demande (facultatif), c'est-à-dire la possibilité d'envoyer du code exécutable depuis le serveur vers le client 
  (lorsqu'il le demande) afin d'étendre les fonctionnalités d'un client.\cite{redhat_rest_spec}
\end{itemize}

Pour ce faire, notre choix s’est porté sur \textbf{Django}, un framework du langage \textbf{Python} offrant une facilité de conception grâce aux nombreuses fonctionnalités déjà incluses par défaut. 
Avec l’emploi de modules comme \textbf{Django Rest Framework}, il est possible de concevoir une \acrshort{api} totalement conforme aux recommandations de la spécification \acrshort{rest}.

\begin{figure}[H]
  \centering
  \includegraphics[width=0.25\linewidth]{django-logo-positive}
  \caption{Logo du framework \textbf{Django}}
  \label{fig:django_logo}
\end{figure}


\subsubsection{Frontend}
La conception de l’interface utilisateur a nécessité l’usage des langages \acrshort{html}, \acrshort{css} et \textbf{Typescript}.

\acrshort{html} est un langage de balisage standardisé, qui permet la conception de documents Web. 
Assisté du langage de style \acrshort{css}, il est possible de concevoir une mise en page attrayante favorisant l'expérience utilisateur.

\textbf{Typescript} est un langage conçu au-dessus du langage \textbf{JavaScript}. 
Il vise notamment à améliorer ce dernier en fournissant un système de typage fort. 
Cela permet entre autres de réduire le nombre de bugs qui finissent en production et d'améliorer l'expérience du développeur.

Afin de faciliter l'intégration de tous les outils suscités, nous avons fait recours au framework \textbf{Vue.js}. 
\textbf{Vue} est un framework moderne de conception d’application Web qui se concentre sur le rendu déclaratif et composition de composants. 
Des solutions complémentaires maintenues officiellement, permettent la gestion du routage, de l'état et bien d’autres fonctionnalités comme les \acrshort{pwa}s. 
Au vu des avantages qu’il présente, il correspond parfaitement aux besoins de notre plateforme en termes d’interfaces.

\begin{figure}[h]
  \centering
  \includegraphics[width=0.25\linewidth]{vue-logo}
  \caption{Logo du framework \textbf{Vue.js}}
  \label{fig:vue_logo}
\end{figure}


\subsubsection{Base de données}
Les systèmes de gestion de bases de données sont des éléments clés dans la conception d’applications dynamiques. 
Elles permettent le stockage, le filtrage, la mise à jour et la suppression des données du système. 
On distingue généralement deux grandes familles de base de données: les bases relationnelles et les bases non relationnelles. 
La première préconise l’utilisation d’un schéma fixe représentant la structure de la donnée alors que le seconde permet une flexibilité du schéma et autorise l’insertion de colonnes quelconques.

Notre choix s’est porté vers PostgreSQL, un système de gestion de base de données relationnel. 
C’est d’ailleurs, le seul système Open Source, fournissant des fonctionnalités dignes de concurrencer les maisons d'édition comme Oracle.

\begin{figure}[H]
  \centering
  \includegraphics[width=0.2\linewidth]{postgresql-logo}
  \caption{Logo de \textbf{PostgreSQL}}
  \label{fig:pg_logo}
\end{figure}

\subsubsection{Gestionnaire de files de tâches}
Effectuer des tâches qui demandent des ressources intensives lors de requêtes \acrshort{http}, risque d’en dégrader les performances. 
Pour éviter celà, nous avons recours à \textbf{Celery}, un gestionnaire de files de tâches moderne qui offre une intégration quasi-parfaite avec le framework \textbf{Django}.

\textbf{Celery} supporte un large panel de protocoles de communication pour la planification des tâches et la récupération des résultats. 
Pour simplifier l’architecture, nous nous sommes servi du cache comme relai afin de déclencher des tâches.


\begin{figure}[H]
  \centering
  \includegraphics[width=0.25\linewidth]{celery-logo}
  \caption{Logo de \textbf{Celery}}
  \label{fig:celery_logo}
\end{figure}


\subsubsection{Cache}
En architecture logicielle, le cache est une composante essentielle. 
Il permet d'améliorer les performances du système en gardant une copie 
(en mémoire, à court terme) des données auxquelles les utilisateurs ont précédemment accédé,
sans qu’il n’y ait besoin de reprendre le même processus de traitement de la requête. 
Ceci réduit le temps de réponse mais aussi réduit la pression sur toute l’infrastructure. 
La base de données est moins sollicitée par exemple. Nous avons opté pour Redis, 
une solution Open Source qui utilise la mémoire vive de la machine pour permettre un accès en lecture 
et en écriture très très rapide.


\begin{figure}[H]
  \centering
  \includegraphics[width=0.25\linewidth]{redis-logo}
  \caption{Logo de \textbf{Redis}}
  \label{fig:redis_logo}
\end{figure}

\subsubsection{Service Writepad}
Dans le cadre de la conception d’un service synchronisé d'écriture, nous avons employé l'éditeur populaire \textbf{TipTap}. 
Il s’agit d’un éditeur Open Source offrant un large panel de fonctionnalités dont la collaboration entre utilisateurs. 
Grâce à une documentation extensive, il est d’autant plus facile de l'intégrer à une application.

\begin{figure}[H]
  \centering
  \includegraphics[width=0.2\linewidth]{tiptap-logo-version}
  \caption{Logo de \textbf{TipTap} et version actuelle}
  \label{fig:tiptap_logo_and_version}
\end{figure}

\subsubsection{Service Whiteboard}
Le service Writepad est le clone d’un projet Open source sous licence MIT, dénommé \textbf{whiteboard}. 
Il est consultable a l’adresse URL suivante: \href{https://github.com/cracker0dks/whiteboard}{https://github.com/cracker0dks/whiteboard}. 
Les modifications effectuées touchent principalement l’interface utilisateur mais aussi font omission de fonctionnalités qui ne nous intéressent pas à l'état actuel du prototype. 
La figure de l’encart \ref{fig:whiteboard_demo} en présente l’interface par défaut.


\begin{figure}[H]
  \centering
  \frame{\includegraphics[width=0.75\linewidth]{whiteboard-upstream-demo}}
  \caption{Interface par défaut de \textbf{whiteboard}}
  \label{fig:whiteboard_demo}
\end{figure}


\subsubsection{Serveur WebRTC}
\textbf{Rust} est un langage compilé qui se veut performant, sûr et productif\cite{rust}. 
Le langage peut notamment donner des garanties d'absence d'erreur de segmentation ou de situation de concurrence 
dès l'étape de compilation. De plus, ceci se fait sans ramasse-miettes. Ses performances sont comparables à celles du \textbf {C} ou du \textbf{C++} pour ce qui concerne la vitesse d'exécution. 
Tout cela en fait un choix excellent pour le développement d'applications de réseau. Parmi les solutions Open Source pour la mise en place d’un serveur \acrshort{webrtc}, 
nous avons opté pour \textbf{Mediasoup}. C’est un serveur de relai \acrshort{sfu} qui offre une bibliothèque \textbf{Rust} pour l'intégration dans diverses catégories d'applications. 

\textbf{Actix} est un framework reposant sur le modèle d’acteurs \cite{actor_design_pattern} et écrit en \textbf{Rust}. 
Il offre un framework web (\textbf{actix-web}) qui permet d’implémenter des services web reposant sur le modèle d’acteurs. 
C’est un framework toutes batteries incluses qui supporte nativement bien de fonctionnalités comme le support des Websockets, lu protocole \acrshort{tls} ou encore de la version 2 du protocole \acrshort{http} (HTTP/2). 
Il offre d’excellentes performances, raison pour laquelle nous l’avons choisi, pour la gestion du signalement lors des connexions \acrshort{webrtc}.

Ci-dessous est un extrait du code source de notre prototype \textbf{StudX}, qui présente la syntaxe du langage \textbf{Rust},  et montre l'intégration des divers outils suscités:
\inputminted{rust}{2-partie/main.rs}

\subsection{Outils de développement}
Le tableau \ref{table:dev_tools} présente une liste non exhaustive des outils de développement employés pour la réalisation du prototype.

\begin{table}[H]
  \centering
\begin{tabular}{|l|l|l|}
  \hline
  \multicolumn{3}{|c|}{Outils de développement} \\
  \hline
  \multicolumn{3}{|c|}{Matériel} \\
  \hline
  Nom & \multicolumn{2}{c|}{Description} \\
  \hline
  Laptop Acer ES1-521 & \multicolumn{2}{l|}{Employé pour les tests de la solution} \\
  \hline
  Laptop Lenovo Thinkbook & \multicolumn{2}{l|}{Employé pour le développement et les tests} \\
  \hline
  Smartphone TECNO Spark 8C & \multicolumn{2}{l|}{Smartphone pour les besoins de tests d'accessibilité} \\
  \hline
  Routeur ZTE MF927U & \multicolumn{2}{l|}{Mise en réseau des appareils} \\
  \hline
  \multicolumn{3}{|c|}{Systèmes d'exploitation} \\
  \hline
  Nom & \multicolumn{2}{l|}{Version} \\
  \hline
  Manjaro Linux & \multicolumn{2}{l|}{22.0.4} \\
  \hline
  Fedora & \multicolumn{2}{l|}{36} \\
  \hline
  \multicolumn{3}{|c|}{Logiciels} \\
  \hline
  Nom & Description & Version \\
  \hline
  \acrshort{edi} Jetbrains & \makecell{Suite de développement logiciel} & \makecell{Versions \textbf{Pro} de \\ Pycharm, \\ CLion et \\ WebStorm} \\
  \hline
  neovim & \makecell{Editeur de texte modal} & 0.8 \\
  \hline
  Git, GitHub & Versionnement de code source & \textit{git}: 2.39.2 \\
  \hline
  Docker & \makecell{Outil de conteneurisation d’applications} & 23.0.1 \\
  \hline
\end{tabular}
\caption{Liste non exhaustive des outils employés dans la réalisation de \textbf{StudX}}
\label{table:dev_tools}
\end{table}

\addcontentsline{toc}{section}{Conclusion}
\section*{Conclusion}
Ce chapitre a permis de passer en revue les choix de conception ainsi que les choix techniques effectués
pour la mise en œuvre de notre prototype d’application. 
Ceci pose des fondations robustes à l'implémentation de ladite solution. En suivant les décisions techniques prises, 
nous avons construit notre prototype. 
Le but du chapitre suivant sera de le présenter puis de faire un bilan d'évaluation.


\chapter{Résultats et Perspectives}\label{chap:3}
\addcontentsline{toc}{section}{Introduction}
\section*{Introduction}
Ce chapitre s'attelle à la présentation du prototype de StudX, l’application de communication en temps réel que nous proposons. 
Nous en présenterons les diverses fonctionnalités accompagnées de capture d'écran. 
Puis, au travers d’une discussion, nous en présenterons les limites, les contraintes et les possibilités d’expansion.

\section{Résultats}
\subsection{Authentification}
L'accès à l’application est subordonné à l'authentification de l’utilisateur. 
La figure \ref{fig:proto_auth} en présente l’interface. 
Elle offre la possibilité de se connecter ou de s'inscrire.

\begin{figure}[h]
  \centering
  \frame{\includegraphics[width=0.85\textwidth]{prototype/login}}
  \caption{Page d'authentification de \textbf{StudX}}
  \label{fig:proto_auth}
\end{figure}

\subsection{Calendrier}
Après authentification, l’utilisateur accède au calendrier des divers événements planifiés. 
Il lui est possible de réduire ou d'étendre la vue au jour actuel, aux semaines ou encore aux mois.  
S’il s’agit d’un administrateur ou d’un enseignant, il peut en ajouter de nouveaux.
La figure \ref{fig:proto_calendar_view} présente le calendrier,qui présente tous les programmes du mois courant.

\begin{figure}[h]
  \centering
  \frame{\includegraphics[width=0.85\textwidth]{prototype/calendar-view}}
  \caption{Calendrier des planifications}
  \label{fig:proto_calendar_view}
\end{figure}

S’il dispose des permissions nécessaires, 
l’utilisateur peut ajouter un événement au calendrier en suivant le formulaire que montre la figure \ref{fig:add_event}.


\begin{figure}[h]
  \centering
  \frame{\includegraphics[width=0.85\textwidth]{prototype/add-event-form}}
  \caption{Calendrier des planifications}
  \label{fig:add_event}
\end{figure}

Il est possible d’associer à l'événement un lien d'accès à la session de conférence en ligne. 
Pour y accéder par la suite, les utilisateurs peuvent consulter les détails dudit événement (figure \ref{fig:event_details}).

\begin{figure}[h]
  \centering
  \frame{\includegraphics[width=0.85\textwidth]{prototype/event-detail}}
  \caption{Détails d'un événement}
  \label{fig:event_details}
\end{figure}

\subsection{Sessions en ligne}
Les événements incluant un lien donnent accès à une session en ligne que
peuvent rejoindre tous les participants disposant du lien.

Les figures \ref{fig:single_user} et \ref{fig:many_users} présentent à quoi ressemble l’interface par défaut. 
Elles présentent, en fait, la grille des participants et l’interface de contrôle.


\begin{figure}[h]
  \centering
  \frame{\includegraphics[width=0.85\textwidth]{prototype/user-single-in-room}}
  \caption{Grille des participants avec un seul participant présent}
  \label{fig:single_user}
\end{figure}


\begin{figure}[h]
  \centering
  \frame{\includegraphics[width=0.85\textwidth]{prototype/user-with-participants-in-room}}
  \caption{Grille des participants avec plus d'un utilisateur}
  \label{fig:many_users}
\end{figure}

\newpage
Outre la voix, les participants ont la possibilité d’interagir entre eux via des messages écrits (figure \ref{fig:room_chat}).

\begin{figure}[h]
  \centering
  \frame{\includegraphics[width=0.85\textwidth]{prototype/room-chat}}
  \caption{Messagerie instantannée intégrée à \textbf{StudX}}
  \label{fig:room_chat}
\end{figure}

Plusieurs autres fonctionnalités sont exploitables. 
L’une d’elles est le partage d'écran. Pour illustrer, nous nous sommes servis de deux appareils avec 
l’un faisant le partage, comme le montre les figures \ref{fig:sharing_screen} et \ref{fig:viewing_screen}.

\newpage
\begin{figure}[h]
  \centering
  \frame{\includegraphics[width=0.85\textwidth]{prototype/user-sharing-screen}}
  \caption{Partage d'écran}
  \label{fig:sharing_screen}
\end{figure}

\newpage
\begin{figure}[h]
  \centering
  \frame{\includegraphics[width=0.85\textwidth]{prototype/user-viewing-screen}}
  \caption{Visualisation de l'écran partagé}
  \label{fig:viewing_screen}
\end{figure}

Les participants disposent également d’un whiteboard, 
c'est-à- dire un tableau virtuel, pour effectuer des illustrations. 
Le contenu est synchronisé entre tous les participants. 
La figure 3.10 fait une démonstration de ladite fonction.

\begin{figure}[h]
  \centering
  \frame{\includegraphics[width=0.85\textwidth]{prototype/whiteboard}}
  \caption{Tableau virtuel}
  \label{fig:whiteboard}
\end{figure}

On peut également percevoir sur l’image, les modifications apportées au projet Open Source qui a servi de base au développement de cette fonctionnalité.

L’application dispose également d’un dispositif de notes intégré, que nous qualifions de \textbf{Writepad}. La figure \ref{fig:writepad} la présente.


\begin{figure}[h]
  \centering
  \frame{\includegraphics[width=0.85\textwidth]{prototype/wrritepad}}
  \caption{Outil de note synchronisé}
  \label{fig:writepad}
\end{figure}

Les fonctions suscitées rendent inaccessible la grille des participants. 
Mais il est toujours possible de pourvoir y accéder dans la même section que la messagerie, comme le montre la figure \ref{fig:participants_aside}.


\begin{figure}[h]
  \centering
  \frame{\includegraphics[width=0.85\textwidth]{prototype/participants}}
  \caption{Liste des participants}
  \label{fig:participants_aside}
\end{figure}

Enfin, chaque utilisateur a la possibilité de quitter la réunion. Si par mégarde, il essaie de recharger par exemple, l’onglet, une confirmation est requise (si le navigateur supporte cette fonctionnalité). 
Les figures \ref{fig:confirm_exit} et \ref{fig:exited} en font l'illustration.


\begin{figure}[h]
  \centering
  \frame{\includegraphics[width=0.85\textwidth]{prototype/confirm-exit}}
  \caption{Confirmation de déconnexion}
  \label{fig:confirm_exit}
\end{figure}


\begin{figure}[h]
  \centering
  \frame{\includegraphics[width=0.85\textwidth]{prototype/no-active-call}}
  \caption{Page de redirection après déconnexion}
  \label{fig:exited}
\end{figure}

\subsection{Autres fonctionnalités}
Dans le but d'améliorer l'expérience utilisateur, nous avons jugé utile d’ajouter quelques fonctionnalités outre celles initialement visées. 
Parmi elles figurent le mode sombre et la mise en place d’un tutoriel interactif expliquant les diverses composantes de notre application. 

\begin{figure}[h]
  \centering
  \frame{\includegraphics[width=0.85\textwidth]{prototype/user-onboarding}}
  \caption{Tutoriel interactif d’introduction à StudX}
  \label{fig:onboarding}
\end{figure}


\begin{figure}[h]
  \centering
  \frame{\includegraphics[width=0.85\textwidth]{prototype/wip-dark-mode}}
  \caption{Mode sombre}
  \label{fig:dark_mode}
\end{figure}

Les administrateurs de la plateforme disposent également d’un accès aux paramètres de 
l'organisation qu’ils dirigent et peuvent ainsi ajouter ou retirer des membres (figures \ref{fig:settings_one} et \ref{fig:settings_two}).

\begin{figure}[h]
  \centering
  \frame{\includegraphics[width=0.85\textwidth]{prototype/settings}}
  \caption{Mode sombre}
  \label{fig:settings_one}
\end{figure}

\begin{figure}[h]
  \centering
  \frame{\includegraphics[width=0.85\textwidth]{prototype/settings-end}}
  \caption{Mode sombre}
  \label{fig:settings_two}
\end{figure}

\newpage
\section{Perspectives}
Le prototype StudX présente un ensemble de fonctionnalités utiles pour le déroulement de classes virtuelles. 
Toutefois, il présente certaines limites, outre les choix de conception comme l’absence de flux vidéo. 

En effet, la plateforme est conçue pour être accessible aux personnes disposant de toutes leurs facultés. 
Bien que les règles basiques d'accessibilité soient prises en compte, 
elles ne couvrent pas totalement le besoin. Par exemple, les personnes malentendantes 
n’ont pas la capacité de tirer profit des échanges vocaux qui sont effectués entre les divers participants. 
Une fonctionnalité envisagée est l'intégration de modèle de Machine Learning permettant la conversion de 
ces signaux audio en gestuelles dans le langage sourd.

Par ailleurs, dans le cadre d’un prototype, les fonctionnalités sont plutôt restreintes. 
On peut ajouter ou améliorer des fonctions comme la persistance de la messagerie instantanée ou 
encore la gestion des utilisateurs.


\addcontentsline{toc}{section}{Conclusion}
\section*{Conclusion}
Le prototype conçu répond à bien des besoins et couvre un tant soit peu l’ensemble des objectifs visés. 
Toutefois, il est possible de l’améliorer, dans le but d’en faire un plus large usage.

% \include{perspectives}
%%conclusion
\conclusion
L'évolution des méthodes d'enseignement implique l'utilisation croissante des technologies de l'information et de la communication. 
Cette tendance est motivée par la nécessité de s'adapter aux nouveaux besoins d'apprentissage et de compenser les insuffisances logistiques et 
financières auxquelles font face les établissements d'enseignement supérieur. Dans ce contexte, 
le projet auquel nous avons contribué visait à fournir un cadre moderne de communication en temps réel, 
permettant d'explorer de nouvelles possibilités en matière de formation virtuelle. 
Nous sommes convaincus que l'application que nous avons développée ouvre la voie à une large gamme d'opportunités en matière d'exploitation des classes virtuelles, 
et nous espérons que notre travail sera une contribution utile à l'ensemble de la communauté éducative.  
Bien qu'ayant atteint ses objectifs initiaux et répondu aux besoins identifiés, il reste encore de nombreuses possibilités d'extension pour en faire un outil encore plus puissant et adapté aux besoins évolutifs de l'enseignement en ligne. 
Il est donc envisageable d'explorer des pistes de développement pour étendre l'application au-delà de ses fonctions actuelles.

\section*{Perspectives}
Le prototype StudX propose une variété de fonctionnalités intéressantes pour la tenue de classes virtuelles.  
Toutefois, il présente certaines limites, outre les choix délibérés de conception comme l’absence de flux vidéo. 

Bien que des règles d'accessibilité basiques aient été prises en compte, elles ne répondent pas entièrement aux 
besoins des personnes malentendantes, qui ne peuvent pas bénéficier des échanges vocaux entre les participants. 
Pour remédier à cela, il serait envisageable d'intégrer un modèle de Machine Learning pour la conversion des signaux 
audio en gestes du langage des signes.

Par ailleurs, dans le cadre d’un prototype, les fonctionnalités sont plutôt restreintes. 
On pourrait ajouter ou améliorer des fonctions comme la persistance de la messagerie instantanée ou 
encore la gestion des utilisateurs. L'enregistrement des sessions pour un usage ultérieur, par exemple, est une fonctionnalité
dont l'implémentation a démarré (branche \href{https://github.com/tobihans/studx/tree/features/room-recording}{room-recording}) dans le but de permettre une amélioration de l'experience utilisateur.

\addcontentsline{toc}{section}{Conclusion}

% 
\lhead[]{} \rhead[]{} \chead[]{}

%%biblio
\addcontentsline{toc}{chapter}{Bibliographie}
\bibliographystyle{abbrv}
\bibliography{biblio}


%\chapter*{Annexe}\addcontentsline{toc}{chapter}{Annexe}\label{annexe1}

\subsection*{Étapes clés du déroulement de l'attaque}


Nous allons exploiter quelques failles de ce réseau pour effectuer une attaque man in the middle (MITM).\\

Au début, notre machine Windows peut atteindre normalement le routeur R4.
\begin{figure}[H]
    \centering
    \includegraphics[scale=0.8]{images/ping_b4_1}
    \caption{Ping vers le routeur R4 avec succès}
    \label{fig:ping_b4_1}
\end{figure}
Quand on essaie de tracer le chemin vers R4, on constate que la machine passe par le routeur R1 légitime du lien pour atteindre R4
\begin{figure}[H]
    \centering
    \includegraphics{images/tracert_b4_1}
    \caption{Traces du chemin vers R4}
    \label{fig:tracert_b41}
\end{figure}
L'attaquant sur le lien peut alors passer a l'attaque.
Pour effectuer l'attaque MITM on utilisera l'outil fake\_router6, un utilitaire du package d'outils \textbf{the hacker choice}.
Ainsi sur la machine d'attaque, on active en un premier lieu le forwarding pour être transparent et ne pas bloquer le transit des paquets.
\begin{figure}[H]
    \centering
    \includegraphics{images/attk/fwrd_activation}
    \caption{Activation du forwarding des paquets.}
    \label{fig:activ_fwrd}
\end{figure}
Aussi on lance wireshark pour observer le trafic des paquets sur notre interface dans le réseau.\\
-------\\
Puisque tout est prêt nous allons lancer l'attaque.

\begin{figure}[H]
    \centering
    \includegraphics[scale=0.8]{images/attk/lancement_attk_1}
    \caption{Initialisation de l'attaque}
    \label{fig:attk_init_1}
\end{figure}

L'attaque est en cours et l'attaquant s'annonce comme le routeur par défaut du lien
nous allons maintenant vérifier la table des routes de notre machine windows.
\begin{figure}[H]
    \centering
    \includegraphics{images/attk/tableRoutes_windows}
    \caption{Table des routes de la machine victime}
    \label{fig:win_route_table}
\end{figure}
On constate que l'attaquant s'est insère comme passerelle de la victime.
pour confirmer cela reprenons un tracert vers le routeur r4
\begin{figure}[H]
    \centering
    \includegraphics{images/attk/tracert_b4_2}
    \caption{Chemin vers b4 pendant l'attaque.}
    \label{fig:tracert_b42}
\end{figure}
On peut voir clairement que la victime passe par l'attaquant pour atteindre le routeur.\\

A présent nous allons essayer de capturer une information envoyée par la victime.
Pour cela la victime fait un telnet sur le router R4 pour s'y connecter avec les paramètres suivants:\\
password1:\textbf{cisco}\\
password2:\textbf{class}
\begin{figure}[H]
    \centering
    \includegraphics{images/attk/telnet_r4}
    \caption{Connexion telnet au routeur.}
    \label{fig:telnetr4}
\end{figure}

Une fois la connexion réussie, nous allons voir avec wireshark les paquets de connexion et y retrouver les paramètres de connexion.
\begin{figure}[H]
    \centering
    \includegraphics[width=1.0\textwidth]{images/attk/c}
    \includegraphics[width=1.0\textwidth]{images/attk/i}
    \includegraphics[width=1.0\textwidth]{images/attk/s}
    \includegraphics[width=1.0\textwidth]{images/attk/c2}
    \includegraphics[width=1.0\textwidth]{images/attk/o}   
    \caption{Premier paramètre de connexion au routeur R4: \textbf{c-i-s-c-o}}
    \label{fig:param_conn_r4}
\end{figure}
\begin{figure}[H]
    \centering
    \includegraphics[width=1.0\textwidth]{images/attk/param2_c}
    \includegraphics[width=1.0\textwidth]{images/attk/param2_l}
    \includegraphics[width=1.0\textwidth]{images/attk/param2_a}
    \includegraphics[width=1.0\textwidth]{images/attk/param2_s1}
    \includegraphics[width=1.0\textwidth]{images/attk/param2_s2}
    \caption{Second paramètre de connexion au routeur R4: \textbf{c-l-a-s-s}}
    \label{fig:param_conn2}
\end{figure}
Les paramètres on été retrouves donc l'attaque a été un succès!

%\subsection*{Mitigations}
%Pour sécuriser ce réseau afin d'éviter ce genre d'attaque, deux mesures de sécurité peuvent être configurées.
%\begin{itemize}
%    \item le SEND
%    \item le RaGuard
%\end{itemize}



\newpage
\tableofcontents

\end{document}          
