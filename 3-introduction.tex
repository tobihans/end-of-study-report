
\introduction

\thispagestyle{plain} % removes unneeded heading

\section*{Contexte}
Les temps changent et les méthodes d’enseignement également. 
Dans le cadre de la mondialisation, on assiste à l'emploi du numérique, 
avec des méthodes de plus en plus créatives et collaboratives, 
pour une meilleure éducation. Cela reste valable dans le milieu de l’enseignement supérieur.
Il va sans dire que des facteurs comme la pandémie de COVID-19 ou encore, 
l'indisponibilité de cadres de cours adéquats, 
rendent plus urgent le besoin de transitionner 
vers des salles de classe virtuelle, pour répondre aux besoins. 
De ce fait, les technologies de l’information et 
de la communication constituent un atout décisif dans le succès de cette transition.

\section*{Problématique}
L’expansion des cours en ligne est désormais un fait. 
Cela requiert une organisation logistique accrue et un investissement financier pour les entités universitaires. 
Toutefois, l’on note l’emploi de solutions génériques, qui rendent difficile, 
voire impossible l'émulation d’un environnement de classe. 
Pour peu qu’elles soient conformes aux exigences, c’est alors le prix qui peut poser problème. 
C’est la raison d'être de notre projet, qui vise la mise en place d’une application, 
pour répondre au besoin d'interactivité lors des cours en ligne, 
et éliminer les barrières d’ordre logistique et financier, imposées par les solutions génériques.

\section*{Objectifs}
Le principal objectif est la conception de StudX, un prototype d’application SaaS, permettant la tenue de cours en ligne.
En termes de fonctionnalités et buts, il s’agira notamment de pouvoir :
\begin{itemize}
  \item organiser les différentes classes, filières ou promotions des entités en sections bien définies ;
  \item définir le calendrier des cours à tenir ;
  \item organiser des sections d’audio-conférence pour le déroulement des cours ;
  \item mettre en place des fonctionnalités telles le partage d'écran et bien d’autre pour émuler un tant soit peu, 
    un environnement de classe présentiel ;
  \item minimiser les coûts requis dans le cadre de la mise en oeuvre d’une solution de classe virtuelle.
\end{itemize}

\thispagestyle{plain} % removes unneeded heading

\section*{Organisation du document}
Le présent document renferme trois chapitres. 
Dans le premier chapitre, nous ferons une revue de littérature sur le sujet 
et présenterons les généralités sur quelques notions essentielles. 
Le second chapitre relate les méthodes employées pour la conception de notre solution ainsi 
que les outils et matériels utilisés à cette fin. 
Le dernier chapitre sera consacré à la présentation des résultats obtenus, 
des interfaces conçues ainsi que des potentielles insuffisances liées à la solution que nous avons développée.
